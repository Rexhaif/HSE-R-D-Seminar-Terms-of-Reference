\documentclass[a4paper, 12pt, titlepage]{article}
\usepackage[a4paper, margin=1in]{geometry}
\usepackage[T2A]{fontenc}
\usepackage{ucs}
\usepackage{xcolor}
\usepackage{soul}
\usepackage[utf8x]{inputenc}
\usepackage[english, russian]{babel}
\usepackage{hyperref}
\usepackage{amstext}
\pagestyle{plain}

% ТЕХНИЧЕСКОЕ ЗАДАНИЕ НА СОЗДАНИЕ АВТОМАТИЗИРОВАННОЙ СИСТЕМЫ
% ГОСТ 34.602-89
% http://www.nist.ru/hr/doc/gost/34-602-89.htm
\setcounter{section}{0}

\begin{document}

\author{
    Даниил Ларионов \\
    Лилия Казакова \\
    Наталия Киселева
}
\title{Техническое задание на создание системы распознавания именованных сущностей в диалоговых текстах}
\date{15 декабря 2020}
\pagebreak
\maketitle

\tableofcontents
\pagebreak
%===========================================================================
% - DONE
\section{Общие сведения}
Наименование системы -- система распознавания именованных сущностей в диалоговых текстах (далее по тексту -- 
Система).

Плановый срок окончания работы по созданию системы -- май 2021 года. 
Отсюда, время, отведенное на работу -- 6 месяцев.

Результат работы по созданию системы будет оформлен в виде научной статьи и представлен на выбранной исполнителями научной конференции.

%===========================================================================

\section{Назначение и цели создания системы}

\hl{\bf TO DO}
\subsection{Назначение системы}

Система предназначена для автоматизации рутинных бытовых задач, выполняемых 
в данный момент человеком. Объект, который будет подвергнут автоматизации -- 
комната в коммунальной квартире и прилегающие к ней места общего пользования.

Система предназначена для решения перечисленных ниже задач:
\begin{enumerate}
\item задачи управления бытовыми электроприборами по расписанию
\item задачи управления бытовыми электроприборами по запросу от пользователя
\item задачи климатического контроля на объекте автоматизации
\item задачи обеспечения связи с внешним миром на объекте автоматизации
\item задачи энергосбережения на объекте автоматизации
\item задачи контроля доступа на объект автоматизации
\end{enumerate}

\hl{\bf TO DO}
\subsection{Цели создания системы}

Целями создания системы являются:
\begin{enumerate}
\item уменьшение количества рутинных бытовых операций, производимых человеком
\item увеличение безопасности объекта автоматизации
\item уменьшение расхода электроэнергии на объекте автоматизации
\item улучшение качества жизни человека на объекте автоматизации
\item увеличение скорости обмена информацией между компьютерными системами 
на объекте автоматизации и человеком
\end{enumerate}

%===========================================================================

\section{Требования к системе}

%---------------------------------------------------------------------------
\hl{\bf TO DO}
\subsection{Требования к системе в целом}

\subsubsection{Требования к структуре и функционированию системы}

Система <<умный дом>> должна быть централизованной -- должен иметься главный управляющий 
компьютер, он же консоль оператора, и набор различных датчиков и управляющих устройств, 
подключенных к этому главному модулю. Также допускается наличие вспомогательного управляющего 
компьютера, который будет включаться в работу когда главный компьютер неработоспособен вследствие 
неисправности или обновления ПО. Вспомогательный компьютер должен обеспечивать возможность управления 
всеми периферийными устройствами по одному, заранее заданному алгоритму, доступ человека к функциям управления 
системой не требуется и будет невозможен до ввода в строй основного компьютера.

Также допускается (но не требуется) управление системой извне -- через WEB-интерфейс, который будет обеспечивать опять же главный 
управляющий компьютер.

Система <<умный дом>> должна иметь следующие подсистемы:
\begin{enumerate}
\item подсистема управления электрооборудованием
\item подсистема управления системами связи с внешним миром
\item подсистема осуществления интерактивного взаимодействия с пользователем
\item подсистема контроля за наличием людей на объекте автоматизации
\item подсистема контроля за различными параметрами окружающей среды (температура, освещенность)
\end{enumerate}

Подсистема управления электрооборудованием предназначена для управления бытовыми лампами накаливания на 220В, 
исполнительными механизмами системы <<умный дом>> и, возможно, электророзетками. Поскольку работа подсистемы 
будет осуществляться в условиях, опасных для жизни человека -- большие напряжения и токи, необходимо обеспечить 
разделение высоковольтной исполнительной составляющей и низковольтных управляющих элементов. Для этого необходимо 
применять полупроводниковые силовые ключи, реле и гальваническую развязку.

Подсистема управления системами связи с внешним миром предназначена для работы с системами связи пользователя 
с внешним миром. Поскольку, на выбранном объекте автоматизации, такими системами являются системы интернет-общения: 
Skype, Jabber, ICQ и E-mail, то работа данной подсистемы заключается в установлении подходящего статуса пользователя 
в этих системах (зависит от предпочтений пользователя, а также от времени суток), оповещении пользователя о непрочитанных 
сообщениях или пропущенных сеансах связи в этих системах, а также оповещении удаленных и доверенных собеседников об 
отсутствиии пользователя дома/на рабочем месте. Также, в задачи этой подсистемы входит задача оповещения пользователя 
об изменениях на используемых им онлайновых сервисах -- на данном объекте автоматизации этими сервисами являются службы 
Google -- Google Reader, Google Calendar и список ToDO. Соответственно, пользователь должен оповещаться обо всех наступающих 
событиях в Google Calendar, наличии изменений в лентах Google Reader и должен иметь удобный доступ к списку ToDo.

Подсистема осуществления интерактивного взаимодействия с пользователем должна использовать звуковые системы оповещения 
пользователя о различных событиях, требующих его внимания. Также допустимо использование графического интерфейса пользователя 
на дисплее, подключенном к главному компьютеру. Допускается использования сенсорного дисплея, в таком случае графический 
интерфейс должен быть оптимизирован под сенсорные экраны. Возможно использование носимой беспроводной гарнитуры, в качестве 
дополнительной системы звукового оповещения, когда невозможна или нежеллательна работа основной (например после 23.00). Также 
допускается использование, в качестве дополнительных интерфейсов взаимодействия с пользователем, системы распознавания речи и 
системы синтеза речи, систему оповещения пользователя средствами мобильной связи. В случае использования системы синтеза 
речи, синтезируемый голос должен быть женским. В случае оповещения пользователя средствами мобильной связи рекомендуемый 
GMS-модуль -- SIM-300.

Подсистема контроля за наличием людей на объекте автоматизации должна обеспечивать выявление наличия людей на контролируемой 
территории. Учет может вестись как при помощи электронного ключа или RFID-метки (добровольный учет), так и при помощи 
датчиков движения или невидимой сетки из ИК или иных лучей на входе на контролируемую территорию (принудительный учет). Также, 
подсистема должна иметь возможность ограничивать доступ людей на контролируемую территорию -- достигается это путем использования 
замков с электронным управлением. В случае добровольного учета людей на подконтрольной территории, необходимо предусмотреть возможность 
автоматического открытия/закрытия дверей при предъявлении считывающему устройству электронного ключа или RFID-метки. Необходимо 
предусмотреть возможность механического отпирания замков при помощи ключа, в случае отказа электронной системы управления замками. 
Также необходимо предусмотреть автоматическое запирание замков в случае отказа системы управления замками.

Подсистема контроля за различными параметрами окружающей среды должна контролировать температуру внутри и вне объекта автоматизации, 
состояние мест общего пользования (занято/свободно), а также давление вне объекта автоматизации, уровень освещенности внутри и вне 
объекта автоматизации и состояние устройства для подогрева воды (чайник), расположенного в местах общего пользования (температуры воды 
в устройстве $< 100^o C \text{ или } \ge 100^o C$).

В системе <<умный дом>> должно быть три уровня иерархии:
\begin{enumerate}
\item Уровень взаимодействия с пользователем -- сюда входят различные WEB-сервисы, системы распознавания и 
синтеза речи, сенсорные панели и прочее.
\item Управляющий уровень -- сюда входит ядро системы, которое способно принимать решения по работе системы 
без участия пользователя. Также в этот уровень входит система резервирования, если она есть.
\item Исполнительный уровень -- сюда входят различные датчики, исполнительные механизмы, системы сопряжения 
предыдущего уровня с системами интернет-общения и онлайновыми сервисами.
\end{enumerate}

Для информационного обмена между вторым и третьим уровнем системы необходимо применять радиоинтерфейс, предположительно -- 
модули сенсорной сети, работающие по протоколу ZigBee. Для связи первого и второго уровней системы необходимо использование 
одноранговой локальной сети Ethernet на витой паре. В случае использования сенсорных дисплеев допустимо применение интерфейсов, 
имеющихся в наличии в этих дисплеях.

Взаимосвязь создаваемой системы со смежными системами не требуется ввиду их отсутствия. В случае создания смежных систем, способ 
обмена информацией между ними и системой <<умный дом>> -- автоматический, по уже созданной локальной сети.

В системе <<умный дом>> должны быть следующие режимы функционирования:
\begin{enumerate}
\item Обычный режим -- работают все подсистемы <<умного дома>>. Управление по большей части автоматическое, с возможностью 
отдачи команд человеком.
\item Ждущий режим -- включается, когда на объекте автоматизации отсутствуют люди. В этом режиме отключены аудиовизуальные 
оповещения и отключена большая часть электрооборудования, например -- лампы накаливания.
\item Ночной режим -- схож со ждущим режимом, из электрооборудования работает система климат-контроля. Система аудиовизуального 
оповещения может работать для предупреждения пользователя об аварийных ситуациях и в качестве будильника.
\item Аварийный режим -- включается в случае отказа главного управляющего компьютера. Работа резервной системы 
состоит в поддержании текущего режима работы система и предотвращения аварийных ситуаций. Работа с автоматизированными 
подсистемами объекта автоматизации осуществляется лишь по запросу от пользователя.
\end{enumerate}

В системе должно быть два способа проведения диагностики. Первый -- упрощенный, когда система во время своей работы 
следит за работоспособностью используемых в данный момент механизмов и оповещает пользователю о неполадках. 
Также, все события в системе должны записываться в файл журнала событий для дальнейшего анализа, желательно на внешнее 
энергонезависимое запоминающее устройство. 
И второй способ -- останов и полная проверка системы, когда производится проверка каждого элемента системы, независимо от того, 
задействован он в данный момент или нет. Пользователю выводится подробный отчет по результатам проверки и работа системы 
возобновляется.

В дальнейшем, возможно расширение системы на новые места и площади, а таккже ее модернизация более функциональными 
исполнительными механизмами и датчиками. Поэтому необходимо сделать систему как можно более модульной, чтобы было легко 
производить замену устаревших деталей на более новые. Для модернизации программного обеспечения необходимо использовать 
возможность удаленной замены ПО на любом, кроме главного и резервного компьютеров, узле системы.

\hl{\bf TO DO} - not sure if we need it
\subsubsection{Требования к надежности}

Показателем надежности для системы является время ее безотказной работы. Поскольку при выходе из строя 
исполнительного (или исполнительных) устройств система будет способна продолжать выполнять свою функцию 
при помощи оставшихся устройств, то для оценки показателей надежности всей системы будут оцениваться 
показатели надежности главного управляющего компьютера (без учета возможного наличия в системе резервного 
компьютера).

\begin{tabular}{|c|c|c|}
\hline
\textbf{Компонент} & \textbf{MTTF (часы)} & \textbf{Вероятность отказа за год} \\
\hline
Микроконтроллер & 219000 & 0.040 \\
\hline
Память & 40000 & 0.219 \\
\hline
Flash-память & 2000000 & 0.004 \\
\hline
Итого & {} & 0.263 \\
\hline
\end{tabular}

Поскольку микросхемы и прочие электронные компоненты относятся к неремонтопригодным устройствам, то для 
оценки надежности используется показатель MTTF (Mean opeating time to failures) -- среднее время до отказа.

Все показатели надежности брались в приблизительном виде из общедоступных источников.

Получаем среднюю вероятность отказа системы в течение года -- 0.263 и среднее время работы системы между 
отказами $\approx 4$ года.

Среднее время восстановления работоспособности системы (MTTR -- Mean Time To Repair) примем равным 11 дням -- 10 дней 
на доставку почтой России некоторых специфичических компонентов и еще один день на установку компонента системы, установку 
соответствующего ПО (если будет необходимо) и настройку/отладку.

Отсюда, коэффициент готовности (вероятность того, что система в любой момент времени будет в рабочем состоянии):

$$K = \frac{MTBF}{MTBF + MTTR} = 0.993$$

Регламентируются требования к надежности для следующих аварийных ситуаций:
\begin{enumerate}
\item Отказ исполнительного механизма или датчика -- система должна продолжить свою работу без использования 
отказавшего узла. Пользователь должен получить предупреждение об отказавшем узле. Запись об отказавшем узле 
должна быть занесена в журнал событий. Отказавший исполнительный механизм должен быть переведен в состояние, 
безопасное для жизнедеятельности человека. Должна быть предусмотрена возможность ручного управления устройством, 
к которому был подключен исполнительный механизм.
\item Отказ главного управляющего компьютера -- система должна аварийно завершить свою работу. Датчики должны перестать 
передавать данные на главный компьютер и перейти в режим пониженного энергопотребления. Исполнительные механизмы должны 
быть переведены в состояние, безопасное для жизнедеятельности человека, а их управляющие контроллеры -- в режим пониженного 
энергопотребления.
\item Отказ части (или всей) подсистемы взаимодействия с пользователем -- система должна продолжить свою работу в штатном 
режиме. По возможности, пользователь должен быть извещен об отказе оборудования, через оставшиеся части подсистемы взаимодействия 
с пользователем.
\end{enumerate}
\\
\hl{\bf TO DO}
\subsubsection{Требования безопасности}

Для обеспечения безопасности монтажника, наладчика и пользователя при работе с исполнительными устройствами, использующими 
опасные для жизни человека значения напряжений, необходимо использовать средства гальванической развязки. Все высоковольтные цепи 
должны быть скрыты в кожухах или корпусах устройств.

Допустимые уровни освещенности, шумовых нагрузок и т.п. не должны превышать установленных в САНПИН 2.1.2.1002-00.
\\
\hl{\bf TO DO}
\subsubsection{Требования к эксплуатации и техническому обслуживанию}

Система предназначена для эксплуатации в условиях жилого помещения (за исключением отдельных модулей). 
Необходимо периодически проводить диагностику узлов системы не реже чем раз в 2 месяца. Полная диагностика системы 
долдна проводиться не реже чем раз в полгода.

Режим питания для системы -- переменный ток 220 В 50 Гц. Для отдельных модулей допускается батарейное питание.

Запасные изделия и приборы не требуются. Допустимо заказывать комплектующие для системы по мере выхода из строя 
уже установленных комплектующих.

\\
\hl{\bf TO DO}
\subsubsection{Требования к защите информации от несанкционированного доступа}

Необходимо реализовать защищенную передачу данных между исполнительными устройствами и главным компьютером. 
Методы организации защищенной передачи данных -- на усмотрение разработчика.

Для доступа к пользовательскому интерфейсу системы необходима авторизация либо при помощи пары логин/пароль, 
либо при помощи кодового слова и биометрической идентификации, либо при помощи лишь биометрической идентификации.

\\
\hl{\bf TO DO}
\subsubsection{Требования по сохранности информации при авариях}

Рри любых отказах исполнительных модулей и датчиков системы должна быть обеспечена сохранность системного журнала. 
При отказе главного компьютера, а также при потере питания, также должна быть обеспечена сохранность и доступность для 
чтения системного журнала, за исключением случаев разрушения носителя, на который записывается дурнал.

\\
\hl{\bf TO DO}
\subsubsection{Требования по стандартизации и унификации}

Допускается использование стандартизированных деталей в исполнительных механизмах, а также в главном 
управляющем компьютере. Датчики системы и контроллеры исполнительных механизмов должны быть реализованы 
самостоятельно.

Допускается использование готовых программных библиотек для отрисовки графического интерфейса при реализации 
пользовательского интерфейса. Также допускается использование любых, пригодных для использования внутри управляющего 
компьютера, операционных систем и программ для организации взаимодействия с пользователем и управляющими 
механизмами/датчиками.

\\
\hl{\bf TO DO}
\subsubsection{Дополнительные требования}

Оснащение системы устройствами для обучения персонала не требуется. К системе должна прилагаться подробная пользовательская 
документация.

%---------------------------------------------------------------------------
\\
\hl{\bf TO DO}
\subsection{Требования к функциям (задачам), выполняемым системой}

Система должна обеспечивать выполнение перечисленных ниже функций:
\begin{enumerate}
\item в рамках первой и второй задачи -- выполнение функции удаленного включения-отключения бытовых осветительных приборов, 
выполнение функции контроля за 
освещенностью, выполнение функции контроля за наличием людей на объекте автоматизации, выполнение функции ручного управления 
осветительными приборами, с оповещением системы об их текущем состоянии.
\item в рамках третьей задачи -- выполнение функции контроля за состоянием температуры на объекте автоматизации и вне него, 
выполнение функции контроля за наличием людей на объекте.
\item в рамках четвертой задачи -- выполнение функции обеспечения работы систем онлайн-общения внутри системы <<умный дом>>, 
выполнение функции обеспечения работы системы с сервисами Google Accounts.
\item в рамках пятой задачи -- решение задачи управления бытовыми электроприборами, выполнение функции контроля за 
освещенностью, выполнение функции контроля за наличием людей на объекте автоматизации.
\item В рамках шестой задачи -- выполнение функции контроля за наличием людей на объекте автоматизации, выполнение функции 
обеспечения авторизованного доступа на объект автоматизации.
\end{enumerate}

Система должна обеспечивать возможность выполнения перечисленных ниже функций:
\begin{enumerate}
\item в рамках первой и второй задачи -- выполнение функции использования заранее заданного пользователем расписания.
\item в рамках третьей задачи -- выполнение функции по управлению тепловентилятором и вентилятором/кондиционером в зависимости 
от температуры на объекте автоматизации.
\item в рамках четвертой задачи -- выполнение функции обеспечения работы системы с 
GSM-модемом в качестве еще одного средства связи
\item в рамках пятой задачи -- выполнение функции использования заранее заданного пользователем расписания.
\item в рамках шестой задачи -- выполнение функции контроля за местоположением людей на объекте автоматизации.
\end{enumerate}

%---------------------------------------------------------------------------

\\
\hl{\bf TO DO}
\subsection{Требования к видам обеспечения}

В системе рекомендуется применять Assembler и язык выского уровня C для реализации драйверов к применяемым 
электронным компонентам системы. Для написания управляющей программы, которая будет работать на главном 
компьютере, рекомендуется применять интерпретируемые языки программирования (Bash, Perl, Python) -- не нужно собирать компилятор 
или использоваться кросскомпилятор под целевую платформу, редактирование исходных текстов и проверку работоспособности 
управляющей программы можно проводить непосредственно на целевой платформе, существенно лишь наличие интерпретатора. Исходные 
коды всех программ системы должны быть подробно прокомментированы на русском языке.

Применяемое программное обеспечение должно иметь патентную чистоту на территории Российской Федерации и принадлежать к 
открытому программному обеспечению (open source).

В документах, описывающих структуру и функциональность системы допустимо применение английского языка, но основной язык для 
таких документов -- русский. Руководство пользователя должно быть написано исключительно на русском языке, английский рекомендуется 
применять для аббревиатур и непереводимых словосочетаний.

%===========================================================================

\\
\hl{\bf TO DO}
\section{Состав и содержание работ по созданию системы}

Перечень стадий и этапов работ по созданию системы:
\begin{enumerate}
\item Исследование и обоснование создания АС
\item Написание технического задания
\item Разработка, как программной так и аппаратной части, отдельных частей системы, которые в дальнейшем будут широко 
применяться во всей системе
\item Разработка устройств, входящих в исполнительный уровень системы
\item Разработка управляющего программного обеспечения, входящего в управляющий уровень системы
\item Разработка программной и, по необходимости аппаратной, части для уровня взаимодействия с пользователем
\item Пробные запуски системы. Проверка работоспособности всех подсистем, объединенных в единую систему
\item Строительно-монтажные работы
\item Пуско-наладочные работы
\item Ввод системы в строй
\end{enumerate}

\\
\hl{\bf TO DO}
\section{Порядок контроля и приемки системы}

Испытания системы будут проводиться на 9 и 10 этапах работ по созданию системы.

Приемная комиссия -- университетская комиссия, производящая оценку и защиту бакалаврских работ.

\\
\hl{\bf TO DO}
\section{Требования к составу и содержанию работ по подготовке объекта автоматизации к вводу системы в действие}

Перед вводом системы в действие необходимо провести подготовительные работы, по созданию посадочных мест для исполнительных 
устройств и датчиков. Также, необходимо выделить место для главного управляющего компьютера и, по необходимости, для резервного.

\\
\hl{\bf TO DO}
\section{Требования к документированию}

Необходимо разработать следующие виды документации:
\begin{enumerate}
\item Схема функциональной структуры
\item Перечень заданий на разработку специализированных (новых) технических средств
\item Технические задания на разработку специализированных (новых) технических средств 
(по необходимости)
\item Описание программного обеспечения
\item Руководство пользователя
\item Общее описание системы
\end{enumerate}

\\
\hl{\bf TO DO}
\section{Источники разработки}

Зарубежные системы-аналоги, на основе которых разрабатывалось ТЗ:
\begin{enumerate}
\item Отдельные узлы системы <<умный дом>>, разработанные компанией HomeSeer 
(\url{http://www.homeseer.com/}.
\item Система <<Home Control System>>, разработанная компанией Home Automation, Inc 
(\url{http://www.homeauto.com}.
\item Система <<iHabitat>>, разработанная компанией Embisys 
(\url{http://www.embisys.com/projects/ihabitat/})
\end{enumerate}

Отечественные системы-аналоги, на основе которых разрабатывалось ТЗ:
\begin{enumerate}
\item Система <<умный дом>> компании Relcom 
(\url{http://www.re-e.ru})
\item Система <<умный дом>> компании intellecthouse 
(\url{http://www.intelecthouse.ru/})
\item Система <<умный дом>> компании СМАРТ Умный дом 
(\url{http://www.smart-home.spb.ru}
\end{enumerate}

Также использовались результаты научно-исследовательских работ в области использования 
сенсорных сетей в системах <<умный дом>> исследовательского центра CEESAR (Center of 
Excellence for Embedded Systems Applied Research \url{http://www.ceesar.ch})

\end{document}

